\section{Logic}
\subsection{Propositions and Logical Operations}

\textbf{Proposition}: a statement that is either \underline{true} or \underline{false}.


Some examples include "It is raining today" and "$3 \cdot 8 = 20 $".


However, not all statements are propositions, such as "open the door"

\begin{center}
  \begin{tabular}{c|c|c}
    \textbf{Name} & \textbf{Symbol} & \textbf{alternate name} \\
    \hline
    NOT           & $\lnot$         & negation                \\
    AND           & $\land$         & conjunction             \\
    OR            & $\lor$          & disjunction             \\
    XOR           & $\oplus$        & exclusive or            \\
  \end{tabular}
  \qquad
  \begin{tabular}{c|c|c|c|c|c}
    \textbf{$p$} & \textbf{$q$} & \textbf{$\lnot p$} & \textbf{$p \land q$} & \textbf{$p \lor q$} & \textbf{$p \oplus q$} \\
    \hline
    T            & T            & F                  & T                    & T                   & F                     \\
    T            & F            & F                  & F                    & T                   & T                     \\
    F            & T            & T                  & F                    & T                   & T                     \\
    F            & F            & T                  & F                    & F                   & F                     \\
  \end{tabular}
\end{center}

XOR is very useful for encryption and binary arithmetic.

\subsection{Evaluating Compound Propositions}

\begin{align*}
  p & : \text{The weather is bad.}    & p \land q            & : \text{The weather is bad \textit{and} the trip is cancelled}                            \\
  q & : \text{The trip is cancelled.} & p \lor q             & : \text{The weather is bad \textit{or} the trip is cancelled}                             \\
  r & : \text{The trip is delayed.}   & p \land (q \oplus r) & : \text{The weather is bad \textit{and} either the trip is cancelled \textit{or} delayed} \\
\end{align*}

\textbf{Order of Evaluation} $\lnot$, then $\land$, then $\lor$, but parenthesis always help for clarity.

\begin{center}
  Example Truth Table:
  \qquad
  \begin{tabular}{c|c|c|c|c}
    $p$ & $q$ & $p \land q$ & $\lnot q$ & $(p \land q) \oplus \lnot q$ \\
    \hline
    T   & T   & T           & F         & T                            \\
    T   & F   & F           & T         & T                            \\
    F   & T   & F           & F         & F                            \\
    F   & F   & F           & T         & T                            \\
  \end{tabular}
\end{center}

\subsection{Conditional Statements}

\begin{center}
  $p \rightarrow q$ \ where $p$ is the \underline{hypothesis} and $q$ is the \underline{conclusion}
\end{center}

\begin{center}
  \begin{tabular}{c|c}
    Format                     & Terminology    \\
    \hline
    $p \rightarrow q$             & given          \\
    $\lnot q \rightarrow \lnot p$ & contrapositive \\
    $q \rightarrow p$             & converse       \\
    $\lnot p \rightarrow \lnot q$ & inverse        \\
  \end{tabular}
  \qquad
  \begin{tabular}{ccccc}
    given   & $p \rightarrow q$             & $\equiv$ & $\lnot q \rightarrow \lnot p$ & contrapositive \\
    inverse & $\lnot p \rightarrow \lnot q$ & $\equiv$ & $q \rightarrow p$             & converse
  \end{tabular}
\end{center}

\begin{center}
  \begin{tabular}{c|c|c}
    $p$ & $q$ & $p \rightarrow q$ \\
    \hline
    T   & T   & T              \\
    T   & F   & F              \\
    F   & T   & T              \\
    F   & F   & T              \\
  \end{tabular}
  \quad
  \begin{tabular}{c}
    $p$ is a \underline{sufficient} condition for $q$ \\
    $q$ is a \underline{necessary} condition for $p$
  \end{tabular}
  \quad
  \begin{tabular}{c|c}
    Phrase                 & Logic          \\
    \hline
    $q$ if $p$             & $p \rightarrow q$ \\
    $q$ only if $p$        & $q \rightarrow p$ \\
    $q$ if and only if $p$ & $p \iff q$
  \end{tabular}
\end{center}

\begin{center}
  \textbf{Order of Operations}: $p \land q \rightarrow r \equiv (p \land q) \rightarrow r$
\end{center}

\subsection{Logical Equivalence}

\begin{center}
  \textbf{Tautology}: a proposition that is always \underline{true}
  \qquad
  \textbf{Contradiction}: a proposition that is always \underline{false}
\end{center}

\textbf{Logically equivalent}: same truth value regardless of the truth values of their individual propositions

\textbf{DeMorgan's Laws}:
\qquad
\begin{tabular}{c}
  $\lnot (p \lor q) \equiv \lnot p \land \lnot q$ \\
  $\lnot (p \land q) \equiv \lnot p \lor \lnot q$
\end{tabular}

\begin{center}
  \begin{tabular}{c}
    Verbally,                                                                                   \\
    It is not true that the patient has migraines \textit{or} high blood pressure $\equiv$      \\
    $\equiv$ The patient does not have migraines \textit{and} does not have high blood pressure \\
    \hline
    It is not true that the patient has migraines \textit{and} high blood pressure $\equiv$     \\
    $\equiv$ The patient does not have migraines \textit{or} does not have high blood pressure  \\
  \end{tabular}
\end{center}

\subsection{Laws of Propositional Logic}

\begin{center}
  You can use \textbf{substitution} on logically equivalent propositions.
\end{center}

\begin{center}
  \begin{tabular}{r|c|c}
    \textbf{Law Name} & $\lor$ or                                               & $\land$ and                                              \\
    \hline
    Idempotent        & $p \lor p \equiv p$                                     & $p \land p \equiv p$                                     \\
    Associative       & $(p \lor q) \lor r \equiv p \lor (q \lor r)$            & $(p \land q) \land r \equiv p \land (q \land r)$         \\
    Commutative       & $p \lor q \equiv q \lor p$                              & $p \land q \equiv q \land p$                             \\
    Distributive      & $p \lor (q \land r) \equiv (p \lor q) \land (p \lor r)$ & $p \land (q \lor r) \equiv (p \land q) \lor (p \land r)$ \\
    Identity          & $p \lor $ F $\equiv p$                                  & $p \land $ T $\equiv p$                                  \\
    Domination        & $p \lor $ T $\equiv $ T                                 & $p \land $ F $\equiv $ F                                 \\
    Double Negation   & $\lnot \lnot p \equiv p$                                                                                           \\
    Complement        & $p \lor \lnot p \equiv$ T                               & $p \land \lnot p \equiv$ F                               \\
    DeMorgan          & $\lnot (p \lor q) \equiv \lnot p \land \lnot q$         & $\lnot (p \land q) \equiv \lnot p \lor \lnot q$          \\
    Absorption        & $p \lor (p \land q) \equiv p$                           & $p \land (p \lor q) \equiv p$                            \\
    Conditional       & $p \rightarrow q \equiv \lnot p \lor q$                    & $p \iff q \equiv (p \rightarrow q) \land (q \rightarrow p)$
  \end{tabular}
\end{center}

\subsection{Predicates and Quantifiers}

\textbf{Predicate}: a logical statement where truth value is a \underline{function} of a variable.

\begin{center}
  P($x$): $x$ is an even number. \qquad P(5): false \qquad P(2): true
\end{center}

\noindent \textbf{Domain}: the set of all possible values for a variable in a predicate.

\begin{center}
  Ex. $ \mathbb{Z}^+ $ is the set of all positive integers.

  *If domain is not clear from context, it should be given as part of the definition of the predicate.
\end{center}

\noindent \textbf{Quantifier}: converts a predicate to a proposition.
\qquad
\begin{tabular}{c|c|c}
  Quantifier  & Symbol     & Meaning        \\
  \hline
  Universal   & $\forall$ & "for all"      \\
  Existential & $\exists$  & "there exists" \\
\end{tabular}

\begin{center}
  $\exists~ x (x+1 < x)$ is false.
\end{center}

\noindent \textbf{Counter Example}: universally quantified statement where an element in the domain
for which the predicate is false. Useful to prove a $\forall~$ statement false.

\subsection{Quantified Statements}

Consider the two following two predicates:
\begin{align*}
  P(x) & : x \text{ is prime, } x \in \mathbb{Z}^+ \\
  O(x) & : x \text{ is odd}
\end{align*}

\begin{center}
  Proposition made of predicates: \qquad $\exists~ x (\text{P}(x) \land \lnot \text{O}(x))$ \\
  Verbally: there exists a positive integer that is prime but is \underline{not} odd.
\end{center}

\noindent \textbf{Free Variable}: a variable that is free to be any value in the domain.

\noindent \textbf{Bound Variable}: a variable that is bound to a quantifier.

\begin{center}
  \begin{tabular}{rl}
    $\text{P}(x)$: & $x \text{ came to the party}$ \\
    $\text{S}(x)$: & $x \text{ was sick}$          \\
  \end{tabular}
  \qquad
  \begin{tabular}{c}
    $\text{P}(x) \overset{?}{\equiv} \lnot \text{S}(x)$ \\
    $\text{P}(x) \centernot{\equiv} \lnot \text{S}(x)$
  \end{tabular}
  \qquad
  \begin{tabular}{c|ccc}
         & P$(x)$ & S$(x)$ & $\lnot$S$(x)$ \\
    \hline
    Joe  & T      & F      & T             \\
    Theo & F      & T      & F             \\
    Gert & T      & F      & T             \\
    Sam  & F      & F      & T
  \end{tabular}
\end{center}

\subsection{DeMorgan's law for Quantified Statements}


Consider the predicate: F$(x):$ "$x$ can fly", where $x$ is a bird.
According to the DeMorgan Identity for Quantified Statements,

\begin{align*}
  \lnot \forall~ x \text{F}(x)    & \equiv \exists~ x \lnot \text{F}(x)                 \\
  \text{"not every bird can fly} & \equiv \text{"there exists a bird that cannot fly}
\end{align*}

Example using DeMorgan Identities:
\begin{align*}
  \lnot \exists~ x (\text{P}(x) \rightarrow \lnot \text{Q}(x)) & \equiv \forall~ x \lnot (\text{P}(x) \rightarrow \lnot \text{Q}(x))          \\
                                                           & \equiv \forall~ x (\lnot \lnot \text{P}(x) \land \lnot \lnot \text{Q}(x)) \\
                                                           & \equiv \forall~ x (\text{P}(x) \land \text{Q}(x))
\end{align*}

\subsection{Nested Quantifiers}

A logical expression with more than one quantifier that binds different variables in the same predicate
is said to have \textbf{Nested Quantifiers}.

\begin{center}
  \begin{tabular}{c|c}
    Logic                                    & Variable Boundedness     \\
    \hline
    $\forall~ x \exists~ y \text{ P}(x, y)$    & $x$, $y$ bound           \\
    $\forall~ x \text{ P}(x, y)$              & $x$ bound, $y$ free      \\
    $\exists~ x \exists~ y \text{ T}(x, y, z)$ & $x$, $y$ bound, $z$ free \\
  \end{tabular}
  \quad
  \begin{tabular}{c|c}
    Logic                                 & Meaning                              \\
    \hline
    $\forall~ x \forall~ y \text{ M}(x, y)$ & "everyone sent an email to everyone" \\
    $\forall~ x \exists~ y \text{ M}(x, y)$ & "everyone sent an email to someone"  \\
    $\exists~ x \forall~ y \text{ M}(x, y)$ & "someone sent an email to everyone"  \\
    $\exists~ x \exists~ y \text{ M}(x, y)$ & "someone sent an email to someone"   \\
  \end{tabular}
\end{center}

\noindent There is a two-player game analogy for how quantifiers work:

\begin{center}
  \begin{tabular}{c|c|c}
    Player                       & Action                                          & Goal                                       \\
    \hline
    Existential Player $\exists~$ & selects value for existentially-bound variables & tries to make expression \underline{true}  \\
    Universal Player $\forall~$   & selects value for universally-bound variables   & tries to make expression \underline{false}
  \end{tabular}
\end{center}

\noindent Consider the predicate L$(x,y):$ "$x$ likes $y$".
\begin{align*}
  \exists~ x \forall~ y \text{L}(x, y)       & \text{ means "there is a student who likes everyone in the school".}  \\
  \lnot \exists~ x \forall~ y \text{L}(x, y) & \text{ means "there is no student who likes everyone in the school".}
\end{align*}
After applying DeMorgan's Laws,
\begin{align*}
  \forall~ x \exists~ y \lnot \text{L}(x, y) & \text{ means "there is no student who likes everyone in the school".}
\end{align*}

\subsection{More Nested Quantifiers}

M$(x, y):$ "x sent an email to y". Consider $\forall~ x \forall~ y$ M $(x, y)$.
It means that "email sent an email to everyone including themselves".
Using $( x \not = y \rightarrow \text{ M}(x, y))$ can fix this quirk.
\begin{align*}
  \forall~ x \forall~ y (x \not = y \rightarrow \text{ M}(x, y))
  \text{ means "everyone sent an email to everyone else}
\end{align*}

\subsubsection*{Expressing Uniqueness in Quantified Statements}

Consider L($x$): $x$ was late to the meeting. If someone was late to the meeting,
how could you express that that someone was the only person late to the meeting?
You want to express that there is someone where everyone else was not late, which
can be done with
\begin{align*}
  \exists~ x (\text{L}(x) \land \forall~ y (x \not = y \rightarrow \lnot \text{L}(y)))
\end{align*}

\subsubsection*{Moving Quantifiers in Logical Statements}

Consider M($x, y$): "$x$ is married to $y$" and A($x$): "$x$ is an adult".
One way of expressing "For every person $x$, if $x$ is an adult, then
there is a person $y$ to whom $x$ is married to" is by this statement:
\begin{align*}
  \forall~ x (\text{ A}(x) \rightarrow \exists~ \text{ M}(x, y))
\end{align*}
Since $y$ does not appear in A($x$), "$\exists~ y$" can be moved so that it appears
just after the "$\forall~$", resulting with
\begin{align*}
  \forall~ x \exists~ y (\text{ A}(x) \rightarrow \text{ M}(x, y))
\end{align*}
When doing this, keep in mind that $\forall~ x \exists~ y \centernot \equiv \exists~ y \forall~ x$:
\begin{align*}
  \forall~ x \exists~ y & (\text{ A}(x) \rightarrow \text{ M}(x, y)) \text{ means}                                 \\
                      & \text{for every $x$, if $x$ is an adult, there exists $y$ who is married to $x$.}     \\
  \exists~ y \forall~ x & (\text{ A}(x) \rightarrow \text{ M}(x, y)) \text{ means}                                 \\
                      & \text{There exists a $y$, such that every $x$ who is an adult is also married to $y$} \\
\end{align*}

\subsection{Logical Reasoning}

\textbf{Argument}: a sequence of propositions, called \underline{hypothesis}, followed
by a final proposition, called the \underline{conclusion}.

An argument is \textbf{valid} if the conclusion is true whenever the hypothesis
are \underline{all} true, otherwise the argument is \textbf{invalid}.

\begin{center}
  An argument is denoted as:
  \begin{tabular}{c}
    $p_1$     \\
    $p_2$     \\
    $\vdots $ \\
    $p_n$     \\
    \hline
    $\therefore c$
  \end{tabular}
  where
  \begin{tabular}{l}
    $p_1, p_2, \ldots, p_n$ are hypothesis \\
    $c$ is the conclusion
  \end{tabular}
\end{center}

The argument is valid whenever the proposition $(p_1 \land p_1 \land \dots \land p_n) \rightarrow c$ is a tautology.
Additionally, because of the commutative law, hypothesis can be reordered without changing the argument.
\begin{center}
  \begin{tabular}{lcl}
    $p$            & \multirow{3}{*}{$\equiv$} & $p \rightarrow q$ \\
    $p \rightarrow q$ &                           & $p$            \\
    \hhline{-~-}
    $\therefore q$ &                           & $\therefore q$
  \end{tabular}
\end{center}

\subsubsection*{The Form of an Argument}

\begin{center}
  \begin{tabular}{lcl}
    It is raining today.                                            &  & $p$            \\
    If it is raining today, then I will not ride my bike to school. &  & $p \rightarrow q$ \\
    \hhline{-~-}
    $\therefore$ I will not ride my bike to school.                 &  & $\therefore q$
  \end{tabular}

  The argument is \underline{valid} because its form,
  \begin{tabular}{l}
    $p$            \\
    $p \rightarrow q$ \\
    \hline
    $\therefore q$
  \end{tabular}
  is an valid argument.

\end{center}

\begin{center}
  \begin{tabular}{lcl}
    5 is not an even number.                          &  & $p$            \\
    If 5 is an even number, then 7 is an even number. &  & $p \rightarrow q$ \\
    \hhline{-~-}
    $\therefore$ 7 is not an even number.             &  & $\therefore q$
  \end{tabular}

  The argument is \underline{invalid} because its form,
  \begin{tabular}{l}
    $\lnot p$      \\
    $p \rightarrow q$ \\
    \hline
    $\therefore \lnot q$
  \end{tabular}
  is an invalid argument.
\end{center}

\subsection{Rules of Inference with Propositions}

Using truth tables to establish the validity of an argument can become tedious, especially if an argument uses a large number of variables.

\begin{center}
  \begin{tabular}{llrll}
    $p$                   & \multirow{3}{*}{Modus Ponens}   & \quad & $p$                       & \multirow{3}{*}{Conjunction}            \\
    $p \rightarrow q$        &                                 &       & $q$                       &                                         \\
    \hhline{-~~-~}
    $\therefore q$        &                                 &       & $\therefore p \land q$    &                                         \\
    \\
    $\lnot q$             & \multirow{3}{*}{Modus Tollens}  &       & $p \rightarrow q$            & \multirow{3}{*}{Hypothetical Syllogism} \\
    $p \rightarrow q$        &                                 &       & $q \rightarrow r$            &                                         \\
    \hhline{-~~-~}
    $\therefore \lnot p$  &                                 &       & $\therefore p \rightarrow r$ &                                         \\
    \\
                          & \multirow{3}{*}{Addition}       &       & $p \lor q$                & \multirow{3}{*}{Disjunctive Syllogism}  \\
    $p$                   &                                 &       & $\lnot p$                 &                                         \\
    \hhline{-~~-~}
    $\therefore p \lor q$ &                                 &       & $\therefore q$            &                                         \\
    \\
                          & \multirow{3}{*}{Simplification} &       & $p \rightarrow q$            & \multirow{3}{*}{Resolution}             \\
    $p \land q$           &                                 &       & $q \rightarrow r$            &                                         \\
    \hhline{-~~-~}
    $\therefore p$        &                                 &       & $\therefore q \lor r$     &                                         \\
    \\
  \end{tabular}
\end{center}

Example expressed in English:
\begin{center}
  \begin{tabular}{lrl}
    If it is raining or windy or both, the game will be cancelled. &  & $(r \lor w) \rightarrow c$ \\
    The game will not be cancelled                                 &  & $\lnot c$               \\
    \hhline{-~-}
    $\therefore$ It is not windy.                                  &  & $\therefore \lnot w$
  \end{tabular}

  Steps to Solve:
  \begin{align}
     & (r \lor w) \rightarrow c &  & \qquad \text{Hypothesis}          \\
     & \lnot c               &  & \qquad \text{Hypothesis}          \\
     & \lnot (r \lor w)      &  & \qquad \text{Modus Tollens: 1, 2} \\
     & \lnot r \land \lnot w &  & \qquad \text{DeMorgan's Law: 3}   \\
     & \lnot w \land \lnot r &  & \qquad \text{Commutative Law: 4}  \\
     & \lnot w               &  & \qquad \text{Simplification: 5}
  \end{align}
\end{center}

\subsection{Rules of Inference with Quantifiers}

In order to apply the rules of quantified expressions, such as $\forall~ x \lnot (\text{ P}(x) \land \text{ Q}(x))$,
we need to remove the quantifier by plugging in a value from the domain to replace the variable x.

For example:
\begin{center}
  \begin{tabular}{lrl}
    Every employee who received a large bonus works hard. &  & $\forall~ x (\text{B}(x) \rightarrow \text{ H}(x))$ \\
    Linda is an employee at the company.                  &  & Linda $\in x$                                   \\
    Linda received a large bonus.                         &  & B(Linda)                                        \\
    \hhline{-~-}
    $\therefore$ Some employee works hard.                &  & $\therefore \exists~ x \text{ H}(x)$
  \end{tabular}
\end{center}

\textbf{Arbitrary Element}: has no special properties other than those shared by all elements of the domain.

\textbf{Particular Element}: may have special properties that are not shared by all the elements of the domain.
For example, if the domain is the set of all integers, $\mathbb{Z}$, a particular element is 3, because it is odd,
which is not true for all integers.

\begin{center}
  \begin{tabular}{lclc}
    \multicolumn{4}{c}{\textbf{Rules of Inference for Quantified Statements}}                                                                                             \\
    $c$ is an element           & \multirow{3}{*}{Universal Instantiation}  &                                               & \multirow{3}{*}{Existential Instantiation*} \\
    $\forall~ x$ P $(x)$         &                                           & $\exists~ x$ P$(x)$                            &                                             \\
    \hhline{-~-~}
    $\therefore$ P$(c)$         &                                           & $\therefore c$ is particular  $\land$ P $(c)$ &                                             \\
    \\
    $c$ is arbitrary            & \multirow{3}{*}{Universal Generalization} & $c$ is an element                             & \multirow{3}{*}{Existential Generalization} \\
    P $(c)$                     &                                           & P$(c)$                                        &                                             \\
    \hhline{-~-~}
    $\therefore \forall~$ P$(x)$ &                                           & $\therefore \exists~ x$ P$(x)$                 &                                             \\
  \end{tabular}
\end{center}

*Each use of Existential Instantiation must define a new element with its own symbol or name.

\subsubsection*{Example of using the Laws of Inference for Quantified Statements}

Consider the following argument:
\begin{tabular}{l}
  $\forall~ x (\text{P}(x) \lor \text{ Q}(x))$ \\
  $3 \text{ is an integer}$                   \\
  $\lnot \text{ P}(3)$                        \\
  \hline
  $\therefore \text{ Q}(3)$
\end{tabular}

\begin{center}
  Steps to Solve:
  \begin{align}
     & \forall~ x (\text{P}(x) \lor \text{ Q}(x)) &  & \qquad \text{Hypothesis}                    \\
     & 3 \text{ is an integer}                   &  & \qquad \text{Hypothesis}                    \\
     & (\text{P}(3) \lor \text{Q}(3))            &  & \qquad \text{Universal Instantiation: 1, 2} \\
     & \lnot \text{ P}(3)                        &  & \qquad \text{ Hypothesis}                   \\
     & \text{ Q}(3)                              &  & \qquad \text{Disjunctive Syllogism: 3, 4}
  \end{align}
\end{center}

\subsubsection*{Showing an Argument with Quantified Statements is Invalid}

Consider the following argument:
\begin{tabular}{l}
  $\exists~ x \text{P}(x)$ \\
  $\exists~ x \text{Q}(x)$ \\
  \hline
  $\therefore \exists~ x (\text{P}(x) \land \text{ Q}(x))$
\end{tabular}

Using a supposed domain \{$c, d$\}, with truth values of
\begin{tabular}{c|cc}
    & P & Q \\
  \hline
  c & T & F \\
  d & F & T
\end{tabular},
the argument is invalid.