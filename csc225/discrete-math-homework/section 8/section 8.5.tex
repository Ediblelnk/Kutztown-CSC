\usepackage[margin=1in]{geometry}
\usepackage{amsmath, amsthm, amssymb, fancyhdr, tikz, circuitikz, graphicx}
\usepackage{centernot, xcolor, hhline, multirow, listings, dashrule}
\usepackage{blkarray, booktabs, bigstrut, etoolbox, extarrows}
\usepackage[normalem]{ulem}
\usepackage{bookmark}
\usetikzlibrary{math}
\usetikzlibrary{fit}

\pagestyle{fancy}

\usepackage{hyperref}
\hypersetup{
  colorlinks=true,
  linkcolor=black,
  filecolor=magenta,
  urlcolor=cyan,
}
%formatting
\newcommand{\bld}{\textbf}
\newcommand{\itl}{\textit}
\newcommand{\uln}{\underline}

%math word symbols
\newcommand{\bb}{\mathbb}
\DeclareMathOperator{\tif}{~\text{if}~}
\DeclareMathOperator{\tand}{~\text{and}~}
\DeclareMathOperator{\tbut}{~\text{but}~}
\DeclareMathOperator{\tor}{~\text{or}~}
\DeclareMathOperator{\tsuchthat}{~\text{such that}~}
\DeclareMathOperator{\tsince}{~\text{since}~}
\DeclareMathOperator{\twhen}{~\text{when}~}
\DeclareMathOperator{\twhere}{~\text{where}~}
\DeclareMathOperator{\tfor}{~\text{for}~}
\DeclareMathOperator{\tthen}{~\text{then}~}
\DeclareMathOperator{\tto}{~\text{to}~}
\DeclareMathOperator{\tin}{~\text{in}~}

%display shortcut
\DeclareMathOperator{\dstyle}{\displaystyle}
\DeclareMathOperator{\sstyle}{\scriptstyle}

%linear algebra
\DeclareMathOperator{\id}{\bld{id}}
\DeclareMathOperator{\vecspan}{\text{span}}
\DeclareMathOperator{\adj}{\text{adj}}

%discrete math - integer properties
\DeclareMathOperator{\tdiv}{\text{div}}
\DeclareMathOperator{\tmod}{\text{mod}}
\DeclareMathOperator{\lcm}{\text{lcm}}

%augmented matrix environment
\newenvironment{apmatrix}[2]{%
  \left(\begin{array}{@{~}*{#1}{c}|@{~}*{#2}{c}}
    }{
  \end{array}\right)
}
\newenvironment{abmatrix}[2]{%
  \left[\begin{array}{@{~}*{#1}{c}|@{~}*{#2}{c}}
      }{
    \end{array}\right]
}

\newenvironment{determinant}[1]{
  \left\lvert
  \begin{array}{@{~}*{#1}{c}}
    }{
  \end{array}
  \right\rvert
}

%lists
\newcommand{\bitem}[1]{\item[\bld{#1.}]}
\newcommand{\bbitem}[2]{\item[\bld{#1.}] \bld{#2}}
\newcommand{\biitem}[2]{\item[\bld{#1.}] \itl{#2}}
\newcommand{\iitem}[1]{\item[\itl{#1.}]}
\newcommand{\iiitem}[2]{\item[\itl{#1.}] \bld{#2}}
\newcommand{\btitem}[2]{\item[\bld{#1.}] \texttt{#2}}

%homework
\newcommand{\question}[2]{\noindent {\large\bld{#1}} #2 \qline}
\newcommand{\qitem}[3]{\item[\bld{#1.}] #2 \qdash \\ #3 \qdash}

\newcommand{\qline}{~\newline\noindent\textcolor[RGB]{200,200,200}{\rule[0.5ex]{\linewidth}{0.2pt}}}
\newcommand{\qdash}{~\newline\noindent\textcolor[RGB]{200,200,200}{\hdashrule[0.5ex]{\linewidth}{0.2pt}{2pt}}}

\newcommand{\assignment}{Section 8.5}

\lhead{Discrete Math II}
\chead{\assignment}
\rhead{Peter Schaefer}

\begin{document}
\section*{\assignment}

\begin{question}{8.5.1}{Proving divisibilty results by induction}
  \qitem{a}{Prove that for any positive integer $n$, $4$ evenly divides $3^{2n}-1$.}
  \begin{proof}
    Base Case: $n=1$
    \[
      3^{2(1)} - 1 = 8 = 4(2)~\checkmark
    \]
    Inductive Hypothesis: Assume that $4$ evenly divides $3^{2k}-1$, for some $k\in \mathbb{Z}^+$. This means that
    \begin{align*}
      3^{2k}-1 & = 4m, \twhere m\in \mathbb{Z}. \\
      3^{2k}   & = 4m+1
    \end{align*}
    Inductive Case: $n = k+1$
    \begin{align*}
      3^{2(k+1)}-1 & = 3^{2k+2}-1                                                 \\
                   & = (3^{2k}\cdot 9) -1                                         \\
                   & = (4m+1)\cdot 9 - 1  &  & \text{by the Inductive Hypothesis} \\
                   & = 36m + 8                                                    \\
                   & = 4(9m+2)
    \end{align*}
    Since $m$ is an integer, $9m+2$ is also an integer. Therefore, $3^{2(k+1)}-1$ is equal to $4$ times an integer. This means that $3^{2(k+1)}-1$ is divisible by $4$. Therefore, for any positive integer $n$, $4$ evenly divides $3^{2n}-1$.
  \end{proof}
  \qitem{c}{Prove that for any positive integer $n$, $4$ evenly divides $11^n-7^n$.}
  \begin{proof}
    Base Case: $n=1$
    \[
      11^1 - 7^1 = 4 = 4(1)~\checkmark
    \]
    Inductive Hypothesis: Assume that $4$ evenly divides $11^k-7^k$, for some $k\in \mathbb{Z}^+$. This means that
    \begin{align*}
      11^k - 7^k & = 4m, \twhere m\in \mathbb{Z}. \\
      11^k       & = 4m + 7^k
    \end{align*}
    Inductive Case: $n = k+1$
    \begin{align*}
      11^{k+1}-7^{k+1} & = 11\cdot 11^k - 7\cdot 7^k                                            \\
                       & = 11\cdot(4m+7^k) - 7\cdot 7^k &  & \text{by the Inductive Hypothesis} \\
                       & = 44m + 4\cdot 7^k                                                     \\
                       & = 4(11m+7^k)
    \end{align*}
    Since $m \tand k$ are both integers, $11m+7^k$ is also an integer. Therefore, $11^{k+1}-7^{k+1}$ is equal to $4$ times an integer, and thus $4$ evenly divides $11^{k+1}-7^{k+1}$. Therefore, for any positive integer $n$, $4$ evenly divides $11^n-7^n$.
  \end{proof}
  \qitem{e}{Prove that for any positive integer $n$, $2$ evenly divides $n^2-5n+2$.}
  \begin{proof}
    Base Case: $n=1$
    \[
      1^2-5(1)+2 = -2 = -2(1)~\checkmark
    \]
    Inductive Hypothesis: Assume that $2$ evenly divides $k^2-5k+2$, for some $k\in\mathbb{Z}^+$. This means that there exists some integer $m$ such that $k^2-5k+2=2m$.

    Inductive Step: $n=k+1$
    \begin{align*}
      (k+1)^2-5(k+1)+2 & = k^2 + 2k + 1 - 5k - 5 + 2                                         \\
                       & = k^2 - 5k + 2 + (2k-4)                                             \\
                       & = 2m + 2(k-2)               &  & \text{by the inductive hypothesis} \\
                       & = 2(m+k-2)
    \end{align*}

    Since $m$ and $k$ are integers, $m+k-2$ is also an integer. Therefore, $(k+1)^2-5(k+1)+2$ is equal to $2$ times an integer, and thus is divisible by $2$. This completes the inductive step.

    By mathematical induction, for any positive integer $n$, $2$ evenly divides $n^2-5n+2$.
  \end{proof}

\end{question}

\begin{question}{8.5.3}{Proving explicit formulas for recurrence relations by induction.}
  \qitem{a}{Define the sequence $\{b_n\}$ as follows:
    \begin{itemize}
      \item $b_0 = 1$
      \item $b_n = 2b_{n-1}+1 \tfor n \geq 1$
    \end{itemize}
    Prove that for $n \geq 0, b_n = 2^{n+1}-1$.}
  \begin{proof}
    Base Case: $n = 0$
    \[
      b_0 = 1 = 2^{0+1}-1~\checkmark
    \]
    Inductive Hypothesis: Assume that for some $k\geq 0$, $b_k=2^{k+1}-1$.

    Inductive Step: $n=k+1$
    \begin{align*}
      b_{k+1} & = 2b_{k}+1                                            \\
              & = 2(2^{k+1}-1)+1 & \text{by the Inductive Hypothesis} \\
              & = 2^{k+2}-2+1                                         \\
              & = 2^{k+2}-1
    \end{align*}
    Therefore, $b_{k+1}=2^{k+2}-1$.

    By mathematical induction, for all $n \geq 0, b_n = 2^{n+1}-1$.
  \end{proof}
\end{question}

\subsection*{8.4.3}
\itl{Prove each of the following statements using mathematical induction.}
\begin{enumerate}
  \biitem{a}{Prove that for $n\geq 2,~3^n > 2^n + n^2$.}
  \begin{proof}
    Base Case: $n = 2$
    \[
      P(2): 3^2 > 2^2 + 2^2 \Rightarrow 9 > 8~\checkmark
    \]
    Inductive Hypothesis: Assume that $P(k)$ is true for some $k \in \bb{Z}^+$ \\
    Inductive Case: $n = k + 1$
    \begin{align*}
      3^{k+1} = 3^k3 & > 3\cdot 2^k + 3\cdot k^2   &  & \text{by inductive hypothesis} \\
                     & > 2\cdot 2^k + k^2 + 2k + 1 &  & \text{since $k>1$}             \\
                     & > 2^{k+1} + (k+1)^2
    \end{align*}
    Therefore $P(k+1)$ is true. Since $P(1)$ is true, and $P(k+1)$ is true, therefore $P(n)$ is true for all $n \in \bb{Z}^+$.
  \end{proof}
\end{enumerate}

\end{document}