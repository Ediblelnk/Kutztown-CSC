\documentclass{article}
\usepackage[margin=1in]{geometry}
\usepackage{amsmath, amsthm, amssymb, fancyhdr, tikz, circuitikz, graphicx}
\usepackage{centernot, xcolor, hhline, multirow, listings, dashrule}
\usepackage{blkarray, booktabs, bigstrut, etoolbox, extarrows}
\usepackage[normalem]{ulem}
\usepackage{bookmark}
\usetikzlibrary{math}
\usetikzlibrary{fit}

\pagestyle{fancy}

\usepackage{hyperref}
\hypersetup{
  colorlinks=true,
  linkcolor=black,
  filecolor=magenta,
  urlcolor=cyan,
}
%formatting
\newcommand{\bld}{\textbf}
\newcommand{\itl}{\textit}
\newcommand{\uln}{\underline}

%math word symbols
\newcommand{\bb}{\mathbb}
\DeclareMathOperator{\tif}{~\text{if}~}
\DeclareMathOperator{\tand}{~\text{and}~}
\DeclareMathOperator{\tbut}{~\text{but}~}
\DeclareMathOperator{\tor}{~\text{or}~}
\DeclareMathOperator{\tsuchthat}{~\text{such that}~}
\DeclareMathOperator{\tsince}{~\text{since}~}
\DeclareMathOperator{\twhen}{~\text{when}~}
\DeclareMathOperator{\twhere}{~\text{where}~}
\DeclareMathOperator{\tfor}{~\text{for}~}
\DeclareMathOperator{\tthen}{~\text{then}~}
\DeclareMathOperator{\tto}{~\text{to}~}
\DeclareMathOperator{\tin}{~\text{in}~}

%display shortcut
\DeclareMathOperator{\dstyle}{\displaystyle}
\DeclareMathOperator{\sstyle}{\scriptstyle}

%linear algebra
\DeclareMathOperator{\id}{\bld{id}}
\DeclareMathOperator{\vecspan}{\text{span}}
\DeclareMathOperator{\adj}{\text{adj}}

%discrete math - integer properties
\DeclareMathOperator{\tdiv}{\text{div}}
\DeclareMathOperator{\tmod}{\text{mod}}
\DeclareMathOperator{\lcm}{\text{lcm}}

%augmented matrix environment
\newenvironment{apmatrix}[2]{%
  \left(\begin{array}{@{~}*{#1}{c}|@{~}*{#2}{c}}
    }{
  \end{array}\right)
}
\newenvironment{abmatrix}[2]{%
  \left[\begin{array}{@{~}*{#1}{c}|@{~}*{#2}{c}}
      }{
    \end{array}\right]
}

\newenvironment{determinant}[1]{
  \left\lvert
  \begin{array}{@{~}*{#1}{c}}
    }{
  \end{array}
  \right\rvert
}

%lists
\newcommand{\bitem}[1]{\item[\bld{#1.}]}
\newcommand{\bbitem}[2]{\item[\bld{#1.}] \bld{#2}}
\newcommand{\biitem}[2]{\item[\bld{#1.}] \itl{#2}}
\newcommand{\iitem}[1]{\item[\itl{#1.}]}
\newcommand{\iiitem}[2]{\item[\itl{#1.}] \bld{#2}}
\newcommand{\btitem}[2]{\item[\bld{#1.}] \texttt{#2}}

%homework
\newcommand{\question}[2]{\noindent {\large\bld{#1}} #2 \qline}
\newcommand{\qitem}[3]{\item[\bld{#1.}] #2 \qdash \\ #3 \qdash}

\newcommand{\qline}{~\newline\noindent\textcolor[RGB]{200,200,200}{\rule[0.5ex]{\linewidth}{0.2pt}}}
\newcommand{\qdash}{~\newline\noindent\textcolor[RGB]{200,200,200}{\hdashrule[0.5ex]{\linewidth}{0.2pt}{2pt}}}

\newcommand{\assignment}{Section 8.12}

\lhead{Discrete Math II}
\chead{\assignment}
\rhead{Peter Schaefer}

\begin{document}
\section*{\assignment}

\question{8.12.1}{Simplify each recurrence reltion as much as possible. The simplified formula for the function should have the same asymptotic growth as the original recurrence relation.}
\begin{enumerate}
  \qitem{a}{$T(n) = T(n-1) + 5n^3 + 4n$}{
    \begin{proof}[work]
      \begin{align*}
        T(n) & = T(n-1) + 5n^3 + 4n                                                       \\
             & = T(n-1) + \Theta(n^3)                                                     \\
             & = \Theta(n^3) + \Theta(n^3) + \cdots + \Theta(n^3) &  & \text{[$n$ times]} \\
             & = n \cdot \Theta(n^3)                                                      \\
             & = \Theta(n^4)
      \end{align*}
    \end{proof}
  }
  \qitem{b}{$T(n) = T(\left\lfloor n/2\right\rfloor) + T(\left\lceil n/2\right\rceil) + 7n$}{
    \begin{proof}[work]
      \begin{align*}
        T(n) & = T(\left\lfloor n/2\right\rfloor) + T(\left\lceil n/2\right\rceil) + 7n                               \\
             & = 2T(n/2) + \Theta(n)                                                                                  \\
             & = 2(2T(n/4) + \Theta(n)) + \Theta(n)                                                                   \\
             & = \Theta(n) + \Theta(n) + \cdots + \Theta(n)                             &  & \text{($\log_2n$ times)} \\
             & = \log_2n \cdot \Theta(n)                                                                              \\
             & = \Theta(n\log n)
      \end{align*}
    \end{proof}
  }
  \qitem{c}{$T(n) = 3 \cdot T(\left\lfloor n/2\right\rfloor) + 14$}{
    \begin{proof}[work]
      \begin{align*}
        T(n) & = 3 \cdot T(\left\lfloor n/2\right\rfloor) + 14                              \\
             & = 3T(n/2) + \Theta(1)                                                        \\
             & = \Theta(n^{\log_23})                           &  & \text{(master theorem)}
      \end{align*}
    \end{proof}
  }
  \qitem{d}{$T(n) = 3 \cdot T(\left\lceil n/3\right\rceil) + 4n + 6n\log n$}{
    \begin{proof}[work]
      \begin{align*}
        T(n) & = 3 \cdot T(\left\lceil n/3\right\rceil) + 4n + 6n\log n                                    \\
             & = 3 \cdot T(n/3) + \Theta(n\log n)                                                          \\
             & = 3(3T(n/9) + \Theta(n\log n)) + \Theta(n\log n)                                            \\
             & = 9T(n/9) + \Theta(n \log n) + \Theta(n\log n)                                              \\
             & = \Theta(n\log n) + \Theta(n\log n) + \cdot + \Theta(n\log n) &  & \text{($log_3 n$ times)} \\
             & = \log n \cdot \Theta(n \log n)                                                             \\
             & = \Theta(n \log^2n)
      \end{align*}
    \end{proof}
  }
\end{enumerate}

\question{8.12.2}{}
\begin{enumerate}
  \item[\bld{a.}]
    \itl{Give the recurrence relation to describe the asymptotic time complexity of your algorithm to compute the sum of the cubes of the first $n$ positive integers.}
    \begin{lstlisting}
CubeSum(n)
{
  if(n == 1) return(1);
  return(n**3 + CubeSum(n-1));
}
      \end{lstlisting}
    \begin{proof}[work]
      \begin{align*}
        T(n) & = T(n-1) + \Theta(1) \\
        T(1) & = \Theta(1)
      \end{align*}
    \end{proof}
\end{enumerate}

\end{document}