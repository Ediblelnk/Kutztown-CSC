\documentclass{article}
\usepackage[margin=1in]{geometry}
\usepackage{amsmath, amsthm, amssymb, fancyhdr, tikz, circuitikz, graphicx}
\usepackage{centernot, xcolor, hhline, multirow, listings, dashrule}
\usepackage{blkarray, booktabs, bigstrut, etoolbox, extarrows}
\usepackage[normalem]{ulem}
\usepackage{bookmark}
\usetikzlibrary{math}
\usetikzlibrary{fit}

\pagestyle{fancy}

\usepackage{hyperref}
\hypersetup{
  colorlinks=true,
  linkcolor=black,
  filecolor=magenta,
  urlcolor=cyan,
}
%formatting
\newcommand{\bld}{\textbf}
\newcommand{\itl}{\textit}
\newcommand{\uln}{\underline}

%math word symbols
\newcommand{\bb}{\mathbb}
\DeclareMathOperator{\tif}{~\text{if}~}
\DeclareMathOperator{\tand}{~\text{and}~}
\DeclareMathOperator{\tbut}{~\text{but}~}
\DeclareMathOperator{\tor}{~\text{or}~}
\DeclareMathOperator{\tsuchthat}{~\text{such that}~}
\DeclareMathOperator{\tsince}{~\text{since}~}
\DeclareMathOperator{\twhen}{~\text{when}~}
\DeclareMathOperator{\twhere}{~\text{where}~}
\DeclareMathOperator{\tfor}{~\text{for}~}
\DeclareMathOperator{\tthen}{~\text{then}~}
\DeclareMathOperator{\tto}{~\text{to}~}
\DeclareMathOperator{\tin}{~\text{in}~}

%display shortcut
\DeclareMathOperator{\dstyle}{\displaystyle}
\DeclareMathOperator{\sstyle}{\scriptstyle}

%linear algebra
\DeclareMathOperator{\id}{\bld{id}}
\DeclareMathOperator{\vecspan}{\text{span}}
\DeclareMathOperator{\adj}{\text{adj}}

%discrete math - integer properties
\DeclareMathOperator{\tdiv}{\text{div}}
\DeclareMathOperator{\tmod}{\text{mod}}
\DeclareMathOperator{\lcm}{\text{lcm}}

%augmented matrix environment
\newenvironment{apmatrix}[2]{%
  \left(\begin{array}{@{~}*{#1}{c}|@{~}*{#2}{c}}
    }{
  \end{array}\right)
}
\newenvironment{abmatrix}[2]{%
  \left[\begin{array}{@{~}*{#1}{c}|@{~}*{#2}{c}}
      }{
    \end{array}\right]
}

\newenvironment{determinant}[1]{
  \left\lvert
  \begin{array}{@{~}*{#1}{c}}
    }{
  \end{array}
  \right\rvert
}

%lists
\newcommand{\bitem}[1]{\item[\bld{#1.}]}
\newcommand{\bbitem}[2]{\item[\bld{#1.}] \bld{#2}}
\newcommand{\biitem}[2]{\item[\bld{#1.}] \itl{#2}}
\newcommand{\iitem}[1]{\item[\itl{#1.}]}
\newcommand{\iiitem}[2]{\item[\itl{#1.}] \bld{#2}}
\newcommand{\btitem}[2]{\item[\bld{#1.}] \texttt{#2}}

%homework
\newcommand{\question}[2]{\noindent {\large\bld{#1}} #2 \qline}
\newcommand{\qitem}[3]{\item[\bld{#1.}] #2 \qdash \\ #3 \qdash}

\newcommand{\qline}{~\newline\noindent\textcolor[RGB]{200,200,200}{\rule[0.5ex]{\linewidth}{0.2pt}}}
\newcommand{\qdash}{~\newline\noindent\textcolor[RGB]{200,200,200}{\hdashrule[0.5ex]{\linewidth}{0.2pt}{2pt}}}

\lhead{Discrete Math II}
\chead{Section 8.8}
\rhead{Peter Schaefer}

\begin{document}

\section*{Section 8.8}

\subsection*{8.8.1}
\begin{enumerate}
  \biitem{a}{Give a recursive definition for strings of properly nested parentheses and curly braces. For example, \(\{\}\)\{\}\(\) is properly nested but \(\{\)\} is not properly nested. The empty string should not be included in your definition.}
  \begin{itemize}
    \item Basis: $\lambda$ has properly nested parentheses and curly braces.
    \item Recursive rules: If $u$ and $v$ are properly nested sequences of parentheses and curly braces then:
          \subitem{1.} $(u),~\{u\},~(\{u\}), \tand \{(u)\}$ is properly nested.
          \subitem{2.} $uv$ is properly nested.
    \item Exclusion statement: a string is properly nested only if it given in the basis or can be constructed by applying the recursive rules to the string in the basis.
  \end{itemize}
\end{enumerate}

\subsection*{8.8.2}
\itl{Let $A = \{a,b\}$.}
\begin{enumerate}
  \biitem{a}{Give a recursive definition for $A^*$.}
  \begin{itemize}
    \item Basis: $\lambda \in A^*$.
    \item Recursive rules: If $u \in A^*$ then:
          \subitem{1.} $ub \tand ua \in A^*$.
          \subitem{2.} $bu \tand au \in A^*$.
    \item Exclusion statement: a string is in $A^*$ if it given in the basis or can be constructed by applying the recursive rules to the string in the basis.
  \end{itemize}
  \biitem{b}{The set $A^+$ is the set of strings over the alphabet $\{a,b\}$ of length at least 1. That is $A^+ = A^* - \{\lambda\}$. Give a recursive definition for $A^+$.}
  \begin{itemize}
    \item Basis: $a \tand b \in A^+$.
    \item Recursive rules: If $u \in A^+$ then:
          \subitem{1.} $ub \tand ua \in A^+$.
          \subitem{2.} $bu \tand au \in A^+$.
    \item Exclusion statement: a string is in $A^+$ if it given in the basis or can be constructed by applying the recursive rules to the strings in the basis.
  \end{itemize}
  \biitem{c}{Let $S$ be the set of all strings from $A^*$ in which there is no $b$ before an $a$. For example, the strings $\lambda$, $aa$, $bbb$, $aabbbb$ all belong to $S$, but $aabab \not \in S$. Give a recursive definition for the set $S$.}
  \begin{itemize}
    \item Basis: $\lambda \in S$.
    \item Recursive rules: If $u \in S$ then:
          \subitem{1.} $au \in S$.
          \subitem{2.} $ub \in S$.
    \item Exclusion statement: a string is in $S$ if it given in the basis or can be constructed by applying the recursive rules to the string in the basis.
  \end{itemize}
\end{enumerate}

\subsection*{8.8.4}
\itl{Give a recursive definition for each subset of the binary strings. A string $x$ should be in the recursively defined set if and only if $x$ has the property described.}
\begin{enumerate}
  \biitem{a}{The set $S$ consists of all strings with an even number of 1's.}
  \begin{itemize}
    \item Basis: $\lambda \in S$.
    \item Recursive rules: If $u \in S$ then:
          \subitem{1.} $1u1,~11u, \tand u11 \in S$.
          \subitem{2.} $0u \tand u0 \in S$.
    \item Exclusion statement: a string is in $S$ if it given in the basis or can be constructed by applying the recursive rules to the string in the basis.
  \end{itemize}
  \biitem{b}{The set $S$ is the set of all binary strings that are palindromes. A string is a palindrome if it is equal to its reverse. For example, $0110$ and $11011$ are both palindromes.}
  \begin{itemize}
    \item Basis: $\lambda,~0, \tand 1 \in S$.
    \item Recursive rules: If $u \in S$ then:
          \subitem{1.} $1u1 \in S$.
          \subitem{2.} $0u0 \in S$.
    \item Exclusion statement: a string is in $S$ if it given in the basis or can be constructed by applying the recursive rules to the string in the basis.
  \end{itemize}
  \biitem{c}{The set $S$ consists of all strings that have the same number of 0's and 1's.}
  \begin{itemize}
    \item Basis: $\lambda \in S$.
    \item Recursive rules: If $u \tand v \in S$ then:
          \subitem{1.} $1u0 \in S$.
          \subitem{2.} $0u1 \in S$.
          \subitem{3.} $uv \in S$.
    \item Exclusion statement: a string is in $S$ if it given in the basis or can be constructed by applying the recursive rules to the string in the basis.
  \end{itemize}
\end{enumerate}

\end{document}