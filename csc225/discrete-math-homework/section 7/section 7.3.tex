\usepackage[margin=1in]{geometry}
\usepackage{amsmath, amsthm, amssymb, fancyhdr, tikz, circuitikz, graphicx}
\usepackage{centernot, xcolor, hhline, multirow, listings, dashrule}
\usepackage{blkarray, booktabs, bigstrut, etoolbox, extarrows}
\usepackage[normalem]{ulem}
\usepackage{bookmark}
\usetikzlibrary{math}
\usetikzlibrary{fit}

\pagestyle{fancy}

\usepackage{hyperref}
\hypersetup{
  colorlinks=true,
  linkcolor=black,
  filecolor=magenta,
  urlcolor=cyan,
}
%formatting
\newcommand{\bld}{\textbf}
\newcommand{\itl}{\textit}
\newcommand{\uln}{\underline}

%math word symbols
\newcommand{\bb}{\mathbb}
\DeclareMathOperator{\tif}{~\text{if}~}
\DeclareMathOperator{\tand}{~\text{and}~}
\DeclareMathOperator{\tbut}{~\text{but}~}
\DeclareMathOperator{\tor}{~\text{or}~}
\DeclareMathOperator{\tsuchthat}{~\text{such that}~}
\DeclareMathOperator{\tsince}{~\text{since}~}
\DeclareMathOperator{\twhen}{~\text{when}~}
\DeclareMathOperator{\twhere}{~\text{where}~}
\DeclareMathOperator{\tfor}{~\text{for}~}
\DeclareMathOperator{\tthen}{~\text{then}~}
\DeclareMathOperator{\tto}{~\text{to}~}
\DeclareMathOperator{\tin}{~\text{in}~}

%display shortcut
\DeclareMathOperator{\dstyle}{\displaystyle}
\DeclareMathOperator{\sstyle}{\scriptstyle}

%linear algebra
\DeclareMathOperator{\id}{\bld{id}}
\DeclareMathOperator{\vecspan}{\text{span}}
\DeclareMathOperator{\adj}{\text{adj}}

%discrete math - integer properties
\DeclareMathOperator{\tdiv}{\text{div}}
\DeclareMathOperator{\tmod}{\text{mod}}
\DeclareMathOperator{\lcm}{\text{lcm}}

%augmented matrix environment
\newenvironment{apmatrix}[2]{%
  \left(\begin{array}{@{~}*{#1}{c}|@{~}*{#2}{c}}
    }{
  \end{array}\right)
}
\newenvironment{abmatrix}[2]{%
  \left[\begin{array}{@{~}*{#1}{c}|@{~}*{#2}{c}}
      }{
    \end{array}\right]
}

\newenvironment{determinant}[1]{
  \left\lvert
  \begin{array}{@{~}*{#1}{c}}
    }{
  \end{array}
  \right\rvert
}

%lists
\newcommand{\bitem}[1]{\item[\bld{#1.}]}
\newcommand{\bbitem}[2]{\item[\bld{#1.}] \bld{#2}}
\newcommand{\biitem}[2]{\item[\bld{#1.}] \itl{#2}}
\newcommand{\iitem}[1]{\item[\itl{#1.}]}
\newcommand{\iiitem}[2]{\item[\itl{#1.}] \bld{#2}}
\newcommand{\btitem}[2]{\item[\bld{#1.}] \texttt{#2}}

%homework
\newcommand{\question}[2]{\noindent {\large\bld{#1}} #2 \qline}
\newcommand{\qitem}[3]{\item[\bld{#1.}] #2 \qdash \\ #3 \qdash}

\newcommand{\qline}{~\newline\noindent\textcolor[RGB]{200,200,200}{\rule[0.5ex]{\linewidth}{0.2pt}}}
\newcommand{\qdash}{~\newline\noindent\textcolor[RGB]{200,200,200}{\hdashrule[0.5ex]{\linewidth}{0.2pt}{2pt}}}

\lhead{Discrete Math II}
\chead{Section 7.3}
\rhead{Peter Schaefer}

\begin{document}

\section*{Section 7.3}

\subsection*{7.3.1}
\begin{lstlisting}
CountValuesLessThanT

Input: a1, a2,...,an
  n, the length of the sequence.
  T, a target value.
Output: The number of values in the sequence that are less than T.

count := 0

For i = 1 to n
  If ( ai < T ), count := count + 1
End-for

Return( count )
\end{lstlisting}
\begin{enumerate}
  \biitem{a}{Characterize the asymptotic growth of the worst-case time complexity of the algorithm. Justify your answer.}
  \begin{proof}
    For any input of size $n$, the loop in the algorithm will execute $n$ times, which is at worst $n$. Therefore the number of operations in the worst case is $cn + d$, which is $\mathcal{O}(n)$.
  \end{proof}
\end{enumerate}

\subsection*{7.3.2}
\begin{lstlisting}
MaximumSubsequenceSum

Input: a1, a2,...,an
  n, the length of the sequence.
Output: The value of the maximum subsequence sum.

maxSum := 0

For i = 1 to n
  thisSum := 0
  For j = i to n
    thisSum := thisSum + aj
    If ( thisSum > maxSum ), maxSum := thisSum
  End-for
End-for

Return( maxSum )
\end{lstlisting}
\begin{enumerate}
  \biitem{a}{Characterize the asymptotic growth of the worst-case time complexity of the algorithm. Justify your answer.}
  \begin{proof}
    For any input of size $n$, the outer loop in the algorithm will execute $n$ times, which is at worst $n$. The inner loop will execute $n-j$ times, which is at worst $n$. Therefore the number of operations in the worst case is $cn^2 + dn + f$, which is $\mathcal{O}(n^2)$.
  \end{proof}
  \biitem{b}{Can you find an algorithm that solves the same problem whose worst-case time complexity is linear?}
  \begin{lstlisting}
MaximumSubsequenceSumLinear

Input: a1, a2,...,an
  n, the length of the sequence.
  Output: The value of the maximum subsequence sum.

maxSum := a1
thisSum := a1

For i = 2 to n
  thisSum := max( ai, thisSum + ai )
  maxSum := max( maxSum, thisSum )
End-for

Return( maxSum )
  \end{lstlisting}
\end{enumerate}

\subsection*{7.3.3}
\begin{lstlisting}
FindMaxFunctionValue

Input: a1, a2,...,an
  n, the length of the sequence.
Output: The largest values of M on input values from the sequence.

max := M(a1, a1, a1)

For i = 1 to n
  For j = 1 to n
    For k = 1 to n
      new := M(ai, aj, ak)
      If ( new > max ), max := new
    End-for
  End-for
End-for

Return( max )
\end{lstlisting}
\begin{enumerate}
  \biitem{a}{Characterize the asymptotic growth of the worst-case time complexity of the algorithm. Justify your answer.}
  \begin{proof}
    For any input of size $n$, the outer loop in the algorithm will execute $n$. The middle loop will execute $n$ times. The innermost loop will execute $n$ times. Therefore the number of operations in the worst case is $an^3 + bn^2 + cn + d$, which is $\mathcal{O}(n^3)$.
  \end{proof}
\end{enumerate}

\end{document}