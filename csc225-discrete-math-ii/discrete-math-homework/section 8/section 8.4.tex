\usepackage[margin=1in]{geometry}
\usepackage{amsmath, amsthm, amssymb, fancyhdr, tikz, circuitikz, graphicx}
\usepackage{centernot, xcolor, hhline, multirow, listings, dashrule}
\usepackage{blkarray, booktabs, bigstrut, etoolbox, extarrows}
\usepackage[normalem]{ulem}
\usepackage{bookmark}
\usetikzlibrary{math}
\usetikzlibrary{fit}

\pagestyle{fancy}

\usepackage{hyperref}
\hypersetup{
  colorlinks=true,
  linkcolor=black,
  filecolor=magenta,
  urlcolor=cyan,
}
%formatting
\newcommand{\bld}{\textbf}
\newcommand{\itl}{\textit}
\newcommand{\uln}{\underline}

%math word symbols
\newcommand{\bb}{\mathbb}
\DeclareMathOperator{\tif}{~\text{if}~}
\DeclareMathOperator{\tand}{~\text{and}~}
\DeclareMathOperator{\tbut}{~\text{but}~}
\DeclareMathOperator{\tor}{~\text{or}~}
\DeclareMathOperator{\tsuchthat}{~\text{such that}~}
\DeclareMathOperator{\tsince}{~\text{since}~}
\DeclareMathOperator{\twhen}{~\text{when}~}
\DeclareMathOperator{\twhere}{~\text{where}~}
\DeclareMathOperator{\tfor}{~\text{for}~}
\DeclareMathOperator{\tthen}{~\text{then}~}
\DeclareMathOperator{\tto}{~\text{to}~}
\DeclareMathOperator{\tin}{~\text{in}~}

%display shortcut
\DeclareMathOperator{\dstyle}{\displaystyle}
\DeclareMathOperator{\sstyle}{\scriptstyle}

%linear algebra
\DeclareMathOperator{\id}{\bld{id}}
\DeclareMathOperator{\vecspan}{\text{span}}
\DeclareMathOperator{\adj}{\text{adj}}

%discrete math - integer properties
\DeclareMathOperator{\tdiv}{\text{div}}
\DeclareMathOperator{\tmod}{\text{mod}}
\DeclareMathOperator{\lcm}{\text{lcm}}

%augmented matrix environment
\newenvironment{apmatrix}[2]{%
  \left(\begin{array}{@{~}*{#1}{c}|@{~}*{#2}{c}}
    }{
  \end{array}\right)
}
\newenvironment{abmatrix}[2]{%
  \left[\begin{array}{@{~}*{#1}{c}|@{~}*{#2}{c}}
      }{
    \end{array}\right]
}

\newenvironment{determinant}[1]{
  \left\lvert
  \begin{array}{@{~}*{#1}{c}}
    }{
  \end{array}
  \right\rvert
}

%lists
\newcommand{\bitem}[1]{\item[\bld{#1.}]}
\newcommand{\bbitem}[2]{\item[\bld{#1.}] \bld{#2}}
\newcommand{\biitem}[2]{\item[\bld{#1.}] \itl{#2}}
\newcommand{\iitem}[1]{\item[\itl{#1.}]}
\newcommand{\iiitem}[2]{\item[\itl{#1.}] \bld{#2}}
\newcommand{\btitem}[2]{\item[\bld{#1.}] \texttt{#2}}

%homework
\newcommand{\question}[2]{\noindent {\large\bld{#1}} #2 \qline}
\newcommand{\qitem}[3]{\item[\bld{#1.}] #2 \qdash \\ #3 \qdash}

\newcommand{\qline}{~\newline\noindent\textcolor[RGB]{200,200,200}{\rule[0.5ex]{\linewidth}{0.2pt}}}
\newcommand{\qdash}{~\newline\noindent\textcolor[RGB]{200,200,200}{\hdashrule[0.5ex]{\linewidth}{0.2pt}{2pt}}}

\lhead{Discrete Math II}
\chead{Section 8.4}
\rhead{Peter Schaefer}

\begin{document}

\section*{Section 8.4}

\subsection*{8.4.1}
\itl{Define P(n) to be the assertion that: $\sum_{j=1}^{n} j^2 = \frac{n(n+1)(2n+1)}{6}$}
\begin{enumerate}
  \biitem{a}{Verify that P(3) is true.}
  \begin{align*}
    P(3): 1^2 + 2^2 + 3^2 & = \frac{3(3+1)(2\cdot 3 + 1)}{6} \\
    1 + 4 + 9             & = \frac{84}{6}                   \\
    14                    & = 14~\checkmark
  \end{align*}
  \biitem{b}{Express P(k).} $P(k): \sum_{j=1}^{k} j^2 = \frac{k(k+1)(2k+1)}{6}$
  \biitem{c}{Express P(k+1).} $P(k): \sum_{j=1}^{k+1} j^2 = \frac{(k+1)(k+2)(2(k+1)+1)}{6}$
  \biitem{d}{In an inductive proof that for every positive integer $n$, $\sum_{j=1}^{n} j^2 = \frac{n(n+1)(2n+1)}{6}$, what must be proven in the base case?} P(1) is true.
  \biitem{e}{In an inductive proof that for every positive integer $n$,$\sum_{j=1}^{n} j^2 = \frac{n(n+1)(2n+1)}{6}$, what must be proven in the inductive step?} If P(k) is true, then P(k+1) is true.
  \biitem{f}{What would be the inductive hypothesis in the inductive step from your previous answer?} Assume that $\sum_{j=1}^{k} j^2 = \frac{k(k+1)(2k+1)}{6}$ is true, for some $k\in \bb{Z}^+$.
  \biitem{g}{Prove by induction that for any positive integer $n$, $\sum_{j=1}^{n} j^2 = \frac{n(n+1)(2n+1)}{6}$.}
  \begin{proof}
    Base Case: $n = 1$
    \[
      P(1): 1^2 = \frac{1(2)(3)}{6} \Rightarrow 1 = 1~\checkmark
    \]
    Inductive Hypothesis: Assume that $P(k)$ is true for some $k \in \bb{Z}^+$ \\
    Inductive Case: $n = k+1$
    \begin{align*}
      \sum_{j=1}^{k+1} j^2 & = \sum_{j=1}^{k} j^2 + (k+1)^2       &  & \text{separating out last term} \\
                           & = \frac{k(k+1)(2k+1)}{6} + (k+1)^2   &  & \text{by inductive hypothesis}  \\
                           & = \frac{k(k+1)(2k+1) + 6(k+1)^2}{6}                                       \\
                           & = \frac{(k+1)[k(2k+1) + 6(k+1)]}{6}                                       \\
                           & = \frac{(k+1)[2k^2 + 7k + 6]}{6}                                          \\
                           & = \frac{(k+1)[(k+2)(2k+3)]}{6}                                            \\
                           & = \frac{(k+1)((k+1)+1)(2(k+1)+1)}{6} &  & \text{by algebra}
    \end{align*}
    Therefore $P(k+1)$ is true. Since $P(1)$ is true, and $P(k+1)$ is true, therefore $P(n)$ is true for all $n \in \bb{Z}^+$.
  \end{proof}

\end{enumerate}

\subsection*{8.4.2}
\itl{Prove each of the following statements using mathematical induction.}
\begin{enumerate}
  \biitem{a}{Prove that for any positive integer $n$, $\sum_{j=1}^{n} j^3 = \left(\frac{n(n+1)}{2}\right)^2$}
  \begin{proof}
    Base Case: $n = 1$
    \[
      P(1): 1^3 = \left(\frac{1(1+1)}{2}\right)^2 \Rightarrow 1 = 1~\checkmark
    \]
    Inductive Hypothesis: Assume that $P(k)$ is true for some $k \in \bb{Z}^+$ \\
    Inductive Case: $n = k + 1$
    \begin{align*}
      \sum_{j=1}^{k+1} j^3 & = \sum_{j=1}^{k} j^3 + (k+1)^3              &  & \text{separating out last term} \\
                           & = \left(\frac{k(k+1)}{2}\right)^2 + (k+1)^3 &  & \text{by inductive hypothesis}  \\
                           & = \frac{k^2(k+1)^2 + 4(k+1)^3}{4}                                                \\
                           & = \frac{(k+1)^2[k^2 + 4k + 4]}{4}                                                \\
                           & = \frac{(k+1)^2(k+2)^2}{4}                                                       \\
                           & = \left(\frac{(k+1)((k+1)+1)}{2}\right)^2   &  & \text{by algebra}
    \end{align*}
    Therefore $P(k+1)$ is true. Since $P(1)$ is true, and $P(k+1)$ is true, therefore $P(n)$ is true for all $n \in \bb{Z}^+$.
  \end{proof}
\end{enumerate}

\subsection*{8.4.3}
\itl{Prove each of the following statements using mathematical induction.}
\begin{enumerate}
  \biitem{a}{Prove that for $n\geq 2,~3^n > 2^n + n^2$.}
  \begin{proof}
    Base Case: $n = 2$
    \[
      P(2): 3^2 > 2^2 + 2^2 \Rightarrow 9 > 8~\checkmark
    \]
    Inductive Hypothesis: Assume that $P(k)$ is true for some $k \in \bb{Z}^+$ \\
    Inductive Case: $n = k + 1$
    \begin{align*}
      3^{k+1} = 3^k3 & > 3\cdot 2^k + 3\cdot k^2   &  & \text{by inductive hypothesis} \\
                     & > 2\cdot 2^k + k^2 + 2k + 1 &  & \text{since $k>1$}             \\
                     & > 2^{k+1} + (k+1)^2
    \end{align*}
    Therefore $P(k+1)$ is true. Since $P(1)$ is true, and $P(k+1)$ is true, therefore $P(n)$ is true for all $n \in \bb{Z}^+$.
  \end{proof}
\end{enumerate}

\end{document}